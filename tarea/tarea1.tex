\documentclass{article}
\usepackage{graphicx}


\begin{document}
\title{Tarea 1}
\author{Yosuá Sal}
\maketitle
\section{Ensayo}
Mi área de interés se basa específicamente en los números en general, ya que si nos damos cuenta
es el idioma universal, como observación me gustaría recalcar que en diferentes países se comunican en
su propia lengua, estos a su vez tienen diferentes maneras de graficar las letras de su abecedario en 
la cual la pronunciación y sonidos varían demasiado, pero por algún motivo la gran mayoría utilizan la 
misma simbología numérica, claro que hay excepciones en las que si usan símbolos diferentes para 
representar los números, pero el valor de estos no cambia, se mantiene continuo y para mí eso es lo 
excepcional de los números, tanta fue la curiosidad y necesidad del ser humano en darle valor a los
objetos que incluso quisieron darle identidad a la nada cuando, la forma de enumerar las cosas era 
bastante concreta.

\subsection{Conseguir que a las personas les llegue a interesar los números y las matemáticas 
para la resolución de problemas en diferentes hambitos de la vida.}

\subsection{Según Gauss: Matemáticas "la reina de las ciencias", constituyendo un lenguaje universal fundamental, 
riguroso y elegante que describía la estructura subyacente del universo, incluyendo la teoría de 
números como su cúspide.}

\subsection{Incorporar al máximo número de personas en en mundo de las matemáticas.}

\subsection{Carl Friedrich Gauss 1,777-1,855.}

    \centering
    \includegraphics[width=0.5\linewidth]{img/logo_ecfm.jpg}
    \label{fig:placeholder}

\end{document}